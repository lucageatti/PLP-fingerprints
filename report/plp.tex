\documentclass[8pt]{article}
\usepackage{amsfonts}
\usepackage{amsmath}
\usepackage{mathtools}
\usepackage{graphicx}
\usepackage{float}
\usepackage{pythonhighlight}
\usepackage[all]{foreign}
\usepackage{cleveref}
\usepackage{fmtcount}
\usepackage{verbatim}
\usepackage{multicol}
\usepackage{xparse}

\NewDocumentCommand\weight{}{\textsf{weight}\xspace}

\title{Probabilistic Logic Programming Exam}
\author{
        Francesco Fabiano, Luca Geatti\\
        \footnotesize Department of Mathematics, Computer Science and Physics \\
        \footnotesize University of Udine\\
        \footnotesize via delle Scienze, 206, 33100 Udine, \underline{Italy}
}
\date{\footnotesize\today}


\begin{document}
\maketitle
\paragraph{Abstract}
This is a report for the exam of the PhD course on Probabilistic Logic
Programming, held at University of Udine in academic year 2018-2019.



\section{Introduction}
\begin{enumerate}
  \item
    Spiegare il problema del fingerprint recognition.
  \item
    Fare un' introduzione su PLP.
  \item
    Spiegare a grandi linee come vogliamo applicare PLP a 
    fingerprint recognition.
\end{enumerate}




%%%%%%%%%%%%%%%%%%%%%%%%%%%%%%%%%%%%%%%%%%%%%%%%%%%%%%%
%%% Encoding
%%%%%%%%%%%%%%%%%%%%%%%%%%%%%%%%%%%%%%%%%%%%%%%%%%%%%%%
\section{Encoding}
\label{sec:encoding}
We start by describing the basic predicates we introduced to model
the problem. In order to model a minutia, we introduced the predicate
  \begin{center}
    \textsf{minutia(X,Y,D,T)}
  \end{center}
where $\textsf{X},\textsf{Y} \in \mathbb{N}$ are the coordinates of the minutiae 
inside the image along the X-axis and Y-axis, respectively;
$\textsf{D} \in [0,2\pi]$ is the direction, expressed in radians; 
and $T \in \{e,b\}$ is the type of the fingerprint ($e$ stands for
\emph{ending}, $b$ for \emph{bifurcation}).
The second predicate is
  \begin{center}
    \textsf{type\_fingerprint(T)}
  \end{center}
where $\textsf{T}\in\{
  \textsf{tented\_archs},
  \textsf{plain\_archs},
  \textsf{left\_loop},
  \textsf{right\_loop},
  \textsf{whorl}
\}$; 
basically, it express which type the overall fingerprint belongs to.
Finally, \textsf{max\_X(N)} and \textsf{max\_Y(N)} express the number
of pixels of the image along the X-axis and the Y-axis, respectively.
The \emph{knowledge base} of our program will be the set of all the 
\textsf{minutia/4} predicates corresponding to the input file,
the \textsf{type\_fingerprint/1} predicate and the \textsf{max\_X/1}
and \textsf{max\_Y/1} predicates.

Since what we want to entail are the edges between the minutiae along
with their probability of existence, we introduced the predicate
  \begin{center}
    \textsf{edge($X_1,Y_1,X_2,Y_2$)}
  \end{center}
where $(X_1,Y_1)$ are the coordinates of one end and $(X_2,Y_2)$
the coordinate of the other end.
Since in our setting we want undirected edges (their direction doesn't
matter), some simmmetry breaking constraints can be added to the 
encoding, aiming at removing specular but equivalent solutions; in
particular, we constrained that there must exists only left-to-right
edges.

\subsection{Structure constraints}
We introduce two structure constraints to forbid certain edges
in the final solution.
\paragraph{Constraint 1}:
each B-minutia has exactly $3$ incident edges. This has been achieved
using the \texttt{aggregate\_all} predicate.
\paragraph{Constraint 2}:
similarly, each E-minutia has exactly $2$ incident edges.


\section{Probability constraints}
\subsection{Tented Archs}
\paragraph{Contraint TA1}:
Given two E-minutiae belonging to a tented arch fingerprint, we note that 
the probability of existence of an edge between these two minutiae 
is higher if they have different directions w.r.t. when they have 
the same direction.

Let $M_1=(X_1,Y_1,D_1,e)$ and $M_2=(X_2,Y_2,D_2,e)$ be two E-minutiae.
Let 
  \begin{align*}
    \weight_d(M_1,M_2) \coloneqq 
    \frac{
      \left( \pi - \Big\lvert \lvert D_1-D_2 \rvert - 
      \pi \rvert\Big\rvert \right)}{\pi}
  \end{align*} 
We give the value $weight_d(M_1,M_2)$ to the probability of existence of
an edge between $M_1$ and $M_2$.



\paragraph{Contraint TA2}:
Given two B-minutiae belonging to a tented arch fingerprint, the more
these two minutiae are distant (either in the x-axis or in the y-axis)
the less an edge between them is likely to exists.

Let $M_1=(X_1,Y_1,D_1,b)$ and $M_2=(X_2,Y_2,D_2,b)$ be two B-minutiae.
Let $max_x$ and $max_y$ be the maximum value for the x-coordinate and
y-coordinate, respectively. We define:
  \begin{align*}
    \weight_x(M_1,M_2) \coloneqq
    \frac{\lvert X_1 - X_2 \rvert}{max_x}
  \end{align*}
  \begin{align*}
    \weight_y(M_1,M_2) \coloneqq
    \frac{\lvert Y_1 - Y_2 \rvert}{max_y}
  \end{align*}
We give the value $\weight_x$ and $\weight_y$ to the probability 
of existence of an edge between $M_1$ and $M_2$.




\subsection{Plain Archs}
Tutti i vincoli uguali a Tented Archs, piu TA5 con ended e tranne "uno meno".


\subsection{Left/Right Loop}
Non dividiamo in 4 quadranti, ma piuttosto diamo un peso a tutte le
regole, cioè sia a quelle standard dei plain archs (due quadranti superiori)
sia a quelle specifiche per i left/right loops (due quadranti inferiori).

I pesi per le probabilità sono opposti per i due tipi di regole, a seconda
che si avvicinino o meno alla parte bassa o alta dell'immagine. L'idea è
questa:
  \begin{itemize}
    \item
      piu mi avvicino alla parte bassa dell'immagine e piu do peso alle
      regole per i left/right loops. Faccio questo \emph{moltiplicando}
      la probabilita dell'arco per il valore:
        \begin{align*}
          1 - Y_1 \cdot Y_2
        \end{align*}
    \item
      similmente, piu mi avvicino alla parte alta dell'immagine e piu 
      do peso alle regole per i plain archs.
  \end{itemize}









\end{document}
