\documentclass[12pt]{article}
\usepackage{amsmath}
\usepackage{mathtools}
\usepackage{graphicx}
\usepackage{float}
\usepackage{pythonhighlight}
\usepackage[all]{foreign}
\usepackage{cleveref}
\usepackage{fmtcount}
\usepackage{verbatim}
\usepackage{multicol}
\usepackage{xparse}

\NewDocumentCommand\weight{}{\textsf{weight}\xspace}

\title{Probabilistic Logic Programming Exam}
\author{
        Francesco Fabiano, Luca Geatti\\
        \footnotesize Department of Mathematics, Computer Science and Physics \\
        \footnotesize University of Udine\\
        \footnotesize via delle Scienze, 206, 33100 Udine, \underline{Italy}
}
\date{\footnotesize\today}


\begin{document}
\maketitle
%\begin{abstract*}
$\dots$
%\end{abstract*}

\section{Encoding}
\label{sec:encoding}

\subsection{Structure constraints}
\paragraph{Constraint 1}:

\paragraph{Constraint 2}:


\section{Probability constraints}
\subsection{Tented Archs}
\paragraph{Contraint TA1}:
Given two E-minutiae belonging to a tented arch fingerprint, we note that 
the probability of existence of an edge between these two minutiae 
is higher if they have different directions w.r.t. when they have 
the same direction.

Let $M_1=(X_1,Y_1,D_1,e)$ and $M_2=(X_2,Y_2,D_2,e)$ be two E-minutiae.
Let 
  \begin{align*}
    \weight_d(M_1,M_2) \coloneqq 
    \frac{
      \left( \pi - \Big\lvert \lvert D_1-D_2 \rvert - 
      \pi \rvert\Big\rvert \right)}{\pi}
  \end{align*} 
We give the value $weight_d(M_1,M_2)$ to the probability of existence of
an edge between $M_1$ and $M_2$.



\paragraph{Contraint TA2}:
Given two B-minutiae belonging to a tented arch fingerprint, the more
these two minutiae are distant (either in the x-axis or in the y-axis)
the less an edge between them is likely to exists.

Let $M_1=(X_1,Y_1,D_1,b)$ and $M_2=(X_2,Y_2,D_2,b)$ be two B-minutiae.
Let $max_x$ and $max_y$ be the maximum value for the x-coordinate and
y-coordinate, respectively. We define:
  \begin{align*}
    \weight_x(M_1,M_2) \coloneqq
    \frac{\lvert X_1 - X_2 \rvert}{max_x}
  \end{align*}
  \begin{align*}
    \weight_y(M_1,M_2) \coloneqq
    \frac{\lvert Y_1 - Y_2 \rvert}{max_y}
  \end{align*}
We give the value $\weight_x$ and $\weight_y$ to the probability 
of existence of an edge between $M_1$ and $M_2$.




\subsection{Plain Archs}
Tutti i vincoli uguali a Tented Archs, piu TA5 con ended e tranne "uno meno".


\subsection{Left/Right Loop}
Non dividiamo in 4 quadranti, ma piuttosto diamo un peso a tutte le
regole, cioè sia a quelle standard dei plain archs (due quadranti superiori)
sia a quelle specifiche per i left/right loops (due quadranti inferiori).

I pesi per le probabilità sono opposti per i due tipi di regole, a seconda
che si avvicinino o meno alla parte bassa o alta dell'immagine. L'idea è
questa:
  \begin{itemize}
    \item
      piu mi avvicino alla parte bassa dell'immagine e piu do peso alle
      regole per i left/right loops. Faccio questo \emph{moltiplicando}
      la probabilita dell'arco per il valore:
        \begin{align*}
          1 - Y_1 \cdot Y_2
        \end{align*}
    \item
      similmente, piu mi avvicino alla parte alta dell'immagine e piu 
      do peso alle regole per i plain archs.
  \end{itemize}









\end{document}
